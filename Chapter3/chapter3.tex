\chapter{Contexte et Problématique}

% **************************** Define Graphics Path **************************
\ifpdf
    \graphicspath{{Chapter3/Figs/Raster/}{Chapter3/Figs/PDF/}{Chapter3/Figs/}}
\else
    \graphicspath{{Chapter3/Figs/Vector/}{Chapter3/Figs/}}
\fi

La gestion de données localisées sur le réseau routier concerne tous les acteurs impliqués dans l’élaboration, la mise en œuvre et l’évaluation des politiques routières. Aujourd’hui, la gestion et l’exploitation du réseau routier se fait :
\begin{itemize}
\item À l'aide de différentes applications informatiques (Bases de données routière, accidents, ouvrages d'art, …) non installées en réseau, non intégrées et sans échanges de données automatisés
\item Au moyen de remontées du terrain manuelles et d'une expertise métier qui ne recourt pas (ou presque pas) à des systèmes automatisés
\end{itemize}
Si toutes ces applications s’appuient sur une description du réseau routier et, dans certains cas, sur des informations de même nature, leur conception et leur utilisation ont été jusqu’ici orientées selon une logique « verticale », privilégiant une approche spécifique par métier.
Leur diffusion au cours des dernières années a conduit les utilisateurs a se constituer des « patrimoines de données individuelles », gérés dans le cadre d’une activité avec peu ou pas d’échanges entre les différentes activités. Ces pratiques tendent en outre a multiplier les bases de données descriptives du réseau dans un même service, générant des surcouts d’acquisition de données, de développement de logiciels et surtout des incohérences d’informations entre les applications routières.