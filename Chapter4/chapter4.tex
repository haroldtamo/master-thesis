\chapter{Objectifs}

% **************************** Define Graphics Path **************************
\ifpdf
    \graphicspath{{Chapter3/Figs/Raster/}{Chapter3/Figs/PDF/}{Chapter3/Figs/}}
\else
    \graphicspath{{Chapter3/Figs/Vector/}{Chapter3/Figs/}}
\fi

Une vision unique et partagée de la représentation du réseau routier est la condition préalable au développement des échanges de données entre tous les acteurs concernés par la connaissance du patrimoine routier.
Du recueil de l’information sur le terrain a son exploitation cartographique dans les applications des différents métiers, le référentiel a pour objectif d’homogénéiser les pratiques de localisation des données routières, pour tous les acteurs de la gestion et de l’exploitation de la route.
Les objectifs de la mise en place d’un système d’information routier peuvent être catégorisés en 2 objectifs principaux : les objectifs stratégiques et les objectifs opérationnels.



Les objectifs stratégiques
\begin{itemize}
\item Améliorer la politique patrimoniale et la sécurité des routes nationales
\item Moderniser les fonctionnalités des outils informatiques actuels, étendre éventuellement leurs champs d'intervention
\item Les intégrer dans une plateforme unique de gestion du réseau routier départemental, constituant un Système d'Information Routière
\item Étoffer leur volet d'aide à la décision, pour optimiser la programmation routière
\end{itemize}



Les objectifs opérationnels
\begin{itemize}
\item Moderniser le suivi et la connaissance en temps réel  
\item Avoir une meilleure connaissance du patrimoine routier 
\item Développer la gestion prévisionnelle des travaux de conservation, de modernisation et de développement du réseau 
\item Développer l’analyse de l’accidentologie
\item Optimiser la gestion du Domaine Public
\item Diffuser au public de l'information routière
\end{itemize}