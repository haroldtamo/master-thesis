%*******************************************************************************
%*********************************** second Chapter *****************************
%*******************************************************************************

\chapter{Etude de l’existant }  %Title of the second Chapter

\ifpdf
    \graphicspath{{Chapter2/Figs/Raster/}{Chapter2/Figs/PDF/}{Chapter2/Figs/}}
\else
    \graphicspath{{Chapter2/Figs/Vector/}{Chapter2/Figs/}}
\fi


%********************************** %First Section  **************************************

Dans cette rubrique il est question de faire un état des lieux en matière de gestion d’éléments constitutifs du système routier en ce moment, De voire les différents travaux déjà réalisé, les différents organes publics et privé interagissant pour la gestion globale du patrimoine routier en matière de normalisation et de production de données. Comme structure recensé et agissant dans la gestion du patrimoine routier nous avons :
\begin{enumerate}
\item  Ministère des transports
\item  Ministères des finances
\item  Ministère des travaux publics
\item  Ministère de la Planification, de la Programmation du développement et de l'Aménagement du Territoire
\item  ministère du Développement Urbain et de l'Habitat
\item  La gendarmerie 
\item  La police
\item  L’institut de la cartographie
\item  CONAROUTE
\item  Fonds routier
\item  Etc…
\end{enumerate}
Toutes ces entités participent à la mise en place des solutions de gestion, de l’ensemble du patrimoine routier sur le plan de la normalisation, de l’amélioration et de la sécurité. 

\section[Présentation des différents acteurs du secteur routier au Cameroun]{Présentation des différents acteurs du secteur routier au Cameroun}
Le dispositif institutionnel de l'Etat du Cameroun prévoit, outre le Premier Ministère, sept départements ministériels charges de la planification, la programmation, la construction, l'entretien et le financement des infrastructures routières nationales. Il s'agit des ministères en charge des
Transports (MINT), de la Planification, de la Programmation du développement et de l'Aménagement du Territoire (MINPLADAT), des Travaux Publics (MINTP), de l'Administration Territoriale et de la Décentralisation (MINATD), du Développement Urbain et de l'Habitat (MINDUH), et des Finances (MINFI).
\begin{itemize}
\item Le Ministère des Transports. 
\end{itemize}
Il est responsable du développement, coordonne tous les modes de transports.
Il assume les missions suivantes relatives au secteur routier :
 L’organisation et le fonctionnement des transports routiers ;
 La coordination des transports routiers ;
 Le suivi de la mise en œuvre et l'exécution du plan sectoriel des transports ;
\begin{itemize}
\item Le Ministère des Travaux Publics (MINTP)
\end{itemize}
L'organisation du Ministère des Travaux Publics est chargée de l'entretien et de la protection du patrimoine routier.
A ce titre, il assume les missions suivantes relatives au secteur routier :
\begin{itemize}
\item[] 
\begin{itemize}
\item il effectue toutes études nécessaires à l'adaptation aux écosystèmes locaux de
Ces infrastructures en liaison avec le Ministère charge de la Recherche
Scientifique, les Institutions de recherche ou d'enseigne ment et de tout autre
Organisme compétent ;
\item il apporte son concours à la construction et à l'entretien des routes, y compris les voiries urbaines, en liaison avec les départements ministériels et organismes compétents, du Parc National de Matériel de Génie Civil (MATGENIE) et du Laboratoire National de Génie Civil (LABOGENIE).
\end{itemize}
\end{itemize}
Les objectifs généraux qui découlent de ces missions sont :
\begin{itemize}
\item[] 
\begin{itemize}
\item Développer et maitriser la gestion du réseau routier national ;
\item Renforcer la protection du patrimoine routier national ;
\item Assurer la maitrise d'œuvre des voiries urbaines ;
\item Assurer le désenclavement de tout le territoire national.
\end{itemize}
\end{itemize}
Pour l'activité routière, le Ministère des Travaux Publics est organisé autour de trois Directions : une Direction des Routes devenue Direction des Investissements et de l'Entretien Routier (en charge des activités de programmation, de préparation et de suivi de la réalisation des opérations de création, d'aménagement, de réhabilitation et d'entretien des routes), une Direction des routes rurales et une Direction des Affaires Générales (en charge de la préparation et de l'exécution du budget). La Direction des Routes comporte les 8 sous directions et cellules suivantes :
\begin{itemize}
\item[] 
\begin{itemize}
\item Une Cellule de la Programmation
\item une Sous-Direction des Investissements Routiers
\item 3 Sous Directions de l'Entretien Routier (réseaux Nord, Sud et Ouest)
\item Une Cellule des Projets de la Banque Africaine de Développement 
\item Une Cellule des Normes et de la Protection du Patrimoine Routier
\end{itemize}
\end{itemize}

\begin{itemize}
\item Le Ministère des Finances (MINFI)
\end{itemize}
Il assume les responsabilités suivantes relatives au secteur routier :
\begin{itemize}
\item[]
\begin{itemize}
\item Il assure la tutelle du Fonds Routier.
Le Fonds Routier a été institué par la Loi n°96/07 du 08 avril 1996 portant protection du patri moine routier national, modifiée et complétée par la Loi n°004/02 du 22 juillet 2004. Un décret présidentiel du 26 aout 1998 fixe ses modalités de fonctionne ment et lui attribue les missions essentielles de suivre les opérations de collecte des ressources, sécuriser les fonds mobilises, contrôler la régularité des contrats et l'éligibilité des dépenses, exécuter rapide ment les paiements. Il s'assure aussi de l'effectivité des travaux d'entretien routier, des prestations de prévention et de sécurité routières, ainsi que la protection du patri moine routier national. Dès l'application de la Loi des Finances de l'exercice 1998/1999, les principales ressources du Fonds Routier
Proviennent de la redevance d'usage de la route (RUR), du péage routier, du pesage routier, des produits des amendes liées à l'usage de la route, des dons et subventions divers.
\end{itemize}
\end{itemize}


